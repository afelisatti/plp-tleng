\textit{G} $=$ $\langle${$V_N$, $V_T$, \textit{P}, BEGIN}$\rangle$ donde 

\indent $V_N$ $=$ \{ BEGIN, HTML, HEAD, LEER\_HEAD, TITLE, SCRIPT, BODY, LEER\_BODY \} 

\indent $V_T$ $=$ \{ initHtml, endHtml, initHead, endHead, initTitle, endTitle, initScript, endScript, initBody, endBody, initDiv, endDiv, initH1, endH1, initP, endP, br, noTags \}

\indent y \textit{P} est\'a dada por: 

  \begin{xlist}{MORE\_SCRIPTS $\longrightarrow$}
    \item[BEGIN $\longrightarrow$] initHtml HTML endHtml
    \item[HTML $\longrightarrow$] HEAD BODY
    \item[HEAD $\longrightarrow$] $\lambda$ $|$ initHead LEER\_HEAD endHead
    \item[LEER\_HEAD $\longrightarrow$] TITLE LEER\_HEAD $|$ \\ SCRIPT LEER\_HEAD $|$ $\lambda$
    \item[TITLE $\longrightarrow$] initTitle noTags endTitle
    \item[SCRIPT $\longrightarrow$] initScript noTags endScripts
    \item[BODY $\longrightarrow$] $\lambda$ $|$ initBody LEER\_BODY endBody
    \item[LEER\_BODY $\longrightarrow$] $\lambda$ $|$ noTags LEER\_BODY $|$ br LEER\_BODY $|$ \\ initDiv LEER\_BODY endDiv LEER\_BODY $|$ \\ initP LEER\_BODY endP LEER\_BODY $|$ \\ initH1 LEER\_BODY endH1 LEER\_BODY  
  \end{xlist}

\indent Para resolver el problema haremos una traducci\'on dirigida por sintaxis incorporando acciones en nuestras producciones.\\
\indent Al comenzar, imprimiremos el comienzo de la estructura HTML que permitir\'a la visualizaci\'on en un \textit{browser}, junto con el estilo que le dar\'a colores a las distintas clases de tags. Estos estar\'an contenidos dentro de elementos \texttt{span} cuya clase fijar\'a los colores. Para la indentaci\'on, los tags que la requieran estar\'an contenidos dentro de bloques \texttt{div}. Esto es basicamente lo sugerido por la c\'atedra.\\
\indent Vale mencionar que nuestros tags ahora ser\'an el texto de otro documento HTML, es decir, el resultado de nuestro programa.