\documentclass[11pt, a4paper, spanish]{article}

%%%%%%%%%% COMIENZO DEL PREAMBULO %%%%%%%%%%

%Info sobre este documento
\author{Martin Cammi}
\title{Trabajo Pr'actico de Teor\'ia de Lenguajes}

%\usepackage{infostyle}                                                  % provee un look & feel similar a un documento Word
\usepackage[top=2.5cm, bottom=2.5cm, left=2.5cm, right=2.5cm]{geometry}  % m\'argenes
\usepackage[ansinew]{inputenc}                                           % permite que los acentos del estilo \'a\'e\'i\'o\'u salgan joya
\usepackage[spanish, activeacute]{babel}                                 % idioma espa\~{n}ol, acentos f\'aciles y deletreo de palabras
\usepackage{indentfirst}                                                 % permite indentar un parrafo a mano
\usepackage{caratula}                                                    % incluye caratula est\'andar
\usepackage{graphicx}                                                    % permite insertar gr\'aficos
\usepackage{color}                                                       % permite el uso de colores en el documento
\usepackage[pdfcreator={TexLive!, LaTeX2e con TeXnicCenter},
			pdfauthor={Grupo 2"},
			pdftitle={Teor\'ia de Lenguajes - Trabajo practico},
			pdfsubject={Trabajo practico: Micro HTML Prettyprint - segunda parte},
			pdfkeywords={MER, MR},
			pdfstartview=FitH,            % Fits the width of the page to the window
			bookmarksnumbered,            % los bookmarks numerados se ven mejor...
			colorlinks,                   % links con bellos colores
			linkcolor=magenta]            % permite cambiar el color de los links
			{hyperref}                    % Permite jugar con algunas cosas que aparecer\'an en el PDF final
\usepackage{hyperref}
\usepackage{ulem}
%\selectlanguage{spanish}
\usepackage{listings}
\usepackage{color}
\usepackage{xcolor}
\usepackage{caption}
\lstset{language=Java,captionpos=t,tabsize=3,frame=no,keywordstyle=\color{blue},
        commentstyle=\color{gray},stringstyle=\color{red},numbers=left,numberstyle=\tiny,
        numbersep=5pt,breaklines=true,showstringspaces=false,basicstyle=\footnotesize,emph={label}}
\linespread{1.3}                    % interlineado equivalente al 1.5 l\'ineas de Word...
\pagestyle{myheadings}              %encabezado personalizable con \markboth{}{}
\markboth{}{Micro HTML PrettyPrint (parser). (Cammi, De Sousa Bispo, Felisatti) }
\headsep = 30pt                     % separaci\'on entre encabezado y comienzo del p\'arrafo

%\addtolength{\oddsidemargin}{-2cm}	% configuracion IDEAL!!!
%\addtolength{\textwidth}{4cm}
%\addtolength{\textheight}{2cm}

% macro 'todo' para To-Do's
\def\todo#1{\textcolor{red}{#1}}

% Macro 'borde' para un texto con borde
\newsavebox{\fmbox}
\newenvironment{borde}[1]
{\begin{lrbox}{\fmbox}\begin{minipage}{#1}}
{\end{minipage}\end{lrbox}\fbox{\usebox{\fmbox}}\\[10pt]}

% para que la gramatica quede linda
\newenvironment{xlist}[1]{%
	\begin{list}{}{%
		\settowidth{\labelwidth}{#1}
		\setlength{\labelsep}{0.5cm}
		\setlength{\leftmargin}{\labelwidth}
		\addtolength{\leftmargin}{\labelsep}
		\setlength{\rightmargin}{0pt}
		\setlength{\parsep}{0.5ex plus0.2ex minus0.1ex}
		\setlength{\itemsep}{0ex plus0.2ex}
		}
	}
{\end{list}}%%%%%%%%%% FIN DEL PREAMBULO %%%%%%%%%%

\begin{document}

\materia{Teor\'ia de Lenguajes}
\submateria{Primer Cuatrimestre de 2013}
\titulo{Trabajo pr\'actico}
\subtitulo{Micro HTML Prettyprint }
\grupo{}

\integrante{Cammi, Mart\'in}{676/02}{martincammi@gmail.com}
\integrante{De Sousa Bispo, Mariano}{389/08}{marian\_sabianaa@hotmail.com}
\integrante{Felisatti, Ana}{335/10}{anafelisatti@gmail.com}


%\begin{center}
%	\includegraphics[scale=0.35]{diagramas/chuck-norris-thumbs-up.png}\\
%\end{center}


\maketitle

\thispagestyle{empty}

\newpage

% Conviene poner las secciones como diferentes archivos,
% sobre todo cuando se trabaja en equipo.
% Es m\'as f\'acil para sincronizar mediante control de versiones.
%\input{Introducci\'on}


% BEGIN Ejemplos de uso

	%\section{Una secci\'on}
	%\label{sec:unaSeccion}
	%Hola! Soy una Secci\'on
	%	\subsection{Una subsecci\'on}
	%		Y yo soy una subsecci\'on!!!
	%		\subsubsection{Una subsubsecci\'on}
	%			Y yo soy una sub-subsecci\'on!!!
	%			\paragraph{Un p\'arrafo\\}
	%				Y yo soy un p\'arrafo, porque no hay mas sub-sub-sub-subsecciones!!!

	%\section{Otra secci\'on}
	%	Como pudimos ver en la secci\'on \ref{sec:unaSeccion}, esto es una demo de una referencia a una secci\'on.
	
	%	Tambi\'en podemos hacer referencia a la p\'agina de la secci\'on:\\[10pt]
	
		% Ejemplo de uso de un borde (falta pulir para que no tire un warning!)
	%	\begin{borde}{0.98\textwidth}
	%		En la p\'agina \pageref{sec:unaSeccion}, hay una secci\'on pilla...
	%	\end{borde}

% END Ejemplos de uso

\section{Introducci\'on}

\indent En el presente trabajo pr\'actico presentaremos un parser para la gram\'atica de la primer parte. Adem\'as, detallaremos las modificaciones realizadas a esta y las desiciones tomadas para implementar el parser. Finalmente, presentaremos una serie de htmls v\'alidos e inv\'alidos. \\
\indent El problema consiste en validar un texto html simple y a su vez obtener una copia visualizable desde un \textit{browser}, con el c\'odigo HTML indentado y con atributos visuales distintivos. Para ello utilizamos la herramienta ANTLR para Java, que nos permiti\'o definir una gr\'amatica con acciones atribuidas a cada producci\'on, junto con los tokens l\'exicos necesarios.

\section{Gram\'atica}
\indent Nuestra gram\'atica, modificada seg\'un las correcciones y modificaciones de implementaci\'on, es la siguiente: \\
	\textit{G} $=$ $\langle${$V_N$, $V_T$, \textit{P}, BEGIN}$\rangle$ donde 

\indent \indent $V_N$ $=$ \{ BEGIN, HTML, HEAD, LEER\_HEAD, TITLE, SCRIPTS, MORE\_SCRIPTS, BODY, LEER\_BODY \} 

\indent \indent $V_T$ $=$ \{ initHtml, endHtml, initHead, endHead, initTitle, endTitle, initScript, endScript, noScripts, initBody, endBody, initDiv, endDiv, initH1, endH1, initP, endP, br, noTags \}

\indent \indent y \textit{P} est\'a dada por: 

  \begin{xlist}{MORE\_SCRIPTS $\longrightarrow$}
    \item[BEGIN $\longrightarrow$] initHtml HTML endHtml
    \item[HTML $\longrightarrow$] HEAD BODY
    \item[HEAD $\longrightarrow$] $\lambda$ $|$ initHead LEER\_HEAD endHead
    \item[LEER\_HEAD $\longrightarrow$] SCRIPTS TITLE MORE\_SCRIPTS $|$ \\ TITLE MORE\_SCRIPTS $|$ MORE\_SCRIPTS
    \item[TITLE $\longrightarrow$] initTitle noTags endTitle
    \item[SCRIPTS $\longrightarrow$] initScript noScripts endScripts MORE\_SCRIPTS
    \item[MORE\_SCRIPTS $\longrightarrow$] $\lambda$ $|$ initBody LEER\_BODY endBody
    \item[BODY $\longrightarrow$] $\lambda$ $|$ initBody LEER\_BODY endBody
    \item[LEER\_BODY $\longrightarrow$] $\lambda$ $|$ noTags LEER\_BODY $|$ br LEER\_BODY $|$ \\ initDiv LEER\_BODY endDiv LEER\_BODY $|$ \\ initP LEER\_BODY endP LEER\_BODY $|$ \\ initH1 LEER\_BODY endH1 LEER\_BODY  
  \end{xlist}


\section{Expresiones regulares}
\indent Nuestros tokens ahora son: \\
	\begin{center}
    \begin{tabular}{| l | l | l | p{5cm} |}
    \hline
    \textbf{\textit{Terminal}} 	& \textbf{\textit{Expresi\'on regular}} 	\\ \hline
    initHtml 		& $<$html$>$ 			 \\ \hline
    endHtml  		& $<$/html$>$  			 \\ \hline
    initHead		& $<$head$>$ 			 \\ \hline
    endHead	 	& $<$/head$>$ 			 \\ \hline
    initBody 		& $<$body$>$ 			 \\ \hline
    endBody		& $<$/body$>$  			 \\ \hline
    initTitle 		& $<$title$>$ 			 \\ \hline
    endTitle  		& $<$/title$>$ 			 \\ \hline
    initScript		& $<$script$>$ 			 \\ \hline
    endScript		& $<$/script$>$  			 \\ \hline
    initDiv		& $<$div$>$ 			 \\ \hline
    endDiv		&  $<$/div$>$ 			 \\ \hline
    initH1		& $<$h1$>$ 			 \\ \hline
    endH1		& $<$/h1$>$  			 \\ \hline
    initP 		& $<$p$>$ 				 \\ \hline
    endP 		& $<$/p$>$ 				 \\ \hline
    br			&$<$br$>$ 				 \\ \hline
    noTags		& \string^((?!(TAGS\_GRAMATICA)).)*\$  			 \\ \hline
    noScripts 		&  \string^((?!($<$script$>\vert<$/script$>$)).)*\$		 \\ \hline
%= ^((?!(<script>$\vert$</script>)).)*$ 

    \end{tabular}
\end{center}

Donde TAGS\_GRAMATICA $=$ $<$html$>$$\vert$$<$/html$>$$\vert$$<$head$>$$\vert$$<$/head$>$$\vert$$<$body$>$$\vert$$<$/body$>$$\vert$ \\
\indent \indent $<$title$>$$\vert$$<$/title$>$$\vert$$<$script$>$$\vert$$<$/script$>$$\vert$$<$div$>$
$\vert$$<$/div$>$$\vert$$<$p$>$$\vert$$<$/p$>$$\vert$$<$h1$>$$\vert$$<$/h1$>$$\vert$$<$br$>$\\

El tag \textit{noTags} se representa con una expresi\'on regular que identifica cualquier texto sin tags definidos por la gram\'atica.\\


Por otro lado, la expresi\'on regular para identificar los comentarios es la siguiente:\\

\indent \indent \textit{comment } $=$ $<$!--(.*?)--$>$\\

Los diferentes s\'imbolos utilizados en las expresiones denotan:
	
  \begin{xlist}{MORE\_SCRIPTS $\longrightarrow$}
    \item[$\vert$ $\longrightarrow$] 	Indica disyunci\'on entre dos expresiones.
    \item[\string^ $\longrightarrow$	]Matchea la expresi\'on al comienzo de una l\'inea. 
    \item[\$ $\longrightarrow$]		Matchea la expresi\'on al final de una l\'inea. 
    \item[?! $\longrightarrow$]		No machea la aparici\'on denotada a su derecha.
    \item[* $\longrightarrow$] 		Indica una repetici\'on de 0 o m\'as veces.
  \end{xlist}

\newpage
\section{Parser}
\indent Utilizando ANTLR, definimos nuestra gram\'atica y los tokens l\'exicos para generar as\'i el lexer y el parser. Su implementaci\'on se encuentra en el archivo \texttt{PrettyPrint.g}. Notar que adem\'as de los tokens previamente nombrados, tenemos los que reconocen espacios y comentarios, ocult\'andolos.\\
\indent La clase \texttt{PrettyPrinter} engloba todas las funciones de impresi\'on que se utilizan en las producciones, por ejemplo, impresi\'on del comienzo de un bloque de indentaci\'on, su final o el comienzo de un \texttt{span} que aporte estilo. \\
\indent Cada producci\'on define lo que debe imprimirse, por ejemplo, sabemos que en 
	 \begin{xlist}{MORE\_SCRIPTS $\longrightarrow$}
	 \item[TITLE $\longrightarrow$] initTitle noTags endTitle
	 \end{xlist}
\indent debemos imprimir la apertura del tag \texttt{title}, el texto en medio y su clasura. \\
\indent Para imprimir el texto utilizamos el atributo \texttt{text} del token \textit{noTags}. Todas estas operaciones tienen un efecto colateral sobre la clase \texttt{PrettyPrinter}, la cual es un Singleton que es accedida por el parser y el main.\\
\indent Para manejar los errores, sobreescribimos el m\'etodo \texttt{reportError} de ANTLR (tanto en el lexer como en el parser) para que levante una excepci\'on y la ejecuci\'on se detenga. De ocurrir tal excepci\'on, es atrapada por el main y se imprime un HTML de error con el mensaje de la misma. 
\section{HTMLs}
\indent A modo de prueba, generamos archivos HTML de entrada que pueden encontrarse en la carpeta \texttt{Tests} adjunta al informe, junto con los \texttt{.html} resultantes para cada uno. A continuaci\'on detallamos que se testeo en cada uno.
\subsection{V\'alidos}
\begin{itemize}
\item{emptyhtml: vac\'io, es decir, sin head ni body.}
\item{emptyheadbody: sin head, \'unicamente con body.}
\item{noscripts: head con s\'olo title, body sin tags internos.}
\item{severalscriptsemptybody: tres scripts con title en medio, sin body.}
\item{nobody: head con title, sin body.}
\item{nestedtags: head con solo scripts, body con varios tags anidados.}
\item{hugebody: title, script, body con varios tags internos y anidamiento.}
\item{comments: sin head, body, dos comentarios.}
\item{input: ejemplo de la c\'atedra.}
\end{itemize}
\subsection{Inv\'alidos}
\begin{itemize}
\item{lexererrorhtml: token inicial incorrecto, es rechazado por el lexer.}
\item{lexererrormismatch: tag en title, es rechazado por el lexer por contener $<$.}
\item{parsererrorunclosedbody: token de cierre de body faltante, es rechazado por el parser.}
\item{parsererrormisplacedtag: token p en head, es rechazado por el parser.}
\end{itemize}

\section{Modo de Uso}
\indent El programa entregado puede ejecutarse en cualquier sistema operativo que soporte Java desde una terminal con el siguiente comando: \texttt{java $\-$jar tleng\_tp2.jar}. Recibe por \texttt{STDIN} el HTML a procesar y devuelve por \texttt{STDOUT} el ya procesado.
\newpage
\section{C\'odigo fuente}
\subsection{Gram\'atica}
\lstinputlisting{PrettyPrint.g}
\newpage
\subsection{Main}
\lstinputlisting{TP2.java}
\newpage
\subsection{Printer}
\lstinputlisting{PrettyPrinter.java}
\newpage

\section{Conclusi\'on}
\indent La utilizaci\'on de gram\'aticas para el parsing de un lenguaje simplifica y abstrae la tarea. ANTLR es una herramienta muy poderosa que permite al desarrollador concentrarse en la aplicaci\'on de la gram\'atica y no en los por menores de la implementaci\'on del parser. Si bien s\'olo hemos explorado una peque\~na parte de lo que las gram\'aticas, expresiones regulares y esta herramienta permiten crear, es evidente su potencial y utilidad.

\end{document}
